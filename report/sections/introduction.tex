\section{Introduction}

Logo is an educational programming language that was developed in 1967
and is commonly used for its simple and intuitive turtle graphics. \cite{wikipediaLogo}
The PrimaLogo project of the University of Basel and ETH Zürich uses
Logo to teach children in the 5th and 6th grades the basics of
programming. To teach, they used the XLogoOnline website, developed
in collaboration with the University of Trier, which allowes users to
write and execute Logo code. The concrete usage of the language is
outlined in the Section \ref{sec:logo-overview}. \\\\
The main problem with the XLogoOnline implementation is that it is
interpreted, which can lead to poor performance. This motivated us
to implement a compiler for the Logo language to improve performance
and to understand how different compilation techniques affected
performance in both compile time and run time.\\\\
Our initial goal was to integrate a compiler into the existing
XLogoOnline implementation, but we abandoned this idea because
it proved too challenging. Instead, we decided to build a standalone
web application that allowes users to write, compile, render, and
benchmark Logo code.\\\\
To compile Logo code, we started with the raw code itself. Our
implementation requires the code to be present in a .lgo file, as
used by XLogoOnline, or it can be written directly in the
application. Files ending in .lgo are essentially text files with
a different extension. This allows the program to easily parse the contents
into a string, which is then converted into an AST using the
grammar, lexer, and parser of XLogoOnline.\\\\
With the AST prepared, we use our compiler to generate a
JavaScript AST, which is later transformed into executable code.
The compiler iterates over the AST and uses the action set, which
contains all built-in functions of the renderer, to convert Logo
AST nodes into JavaScript AST nodes. In some cases, additional
steps are necessary to ensure correct execution; these are
described in the Implementation section.\\\\
The compiler supports different modes for code generation. The
default mode is direct access, where each function call of the
action set uses member expression. The array access mode uses
computed member expression to access the functions of the action set. The direct mode
avoids function calls entirely and instead implements the
functions directly in the code. After generating the JavaScript AST,
we use Escodegen \cite{escodegen} to produce the final JavaScript code, which is
then used for rendering.\\\\
To render the code, the application uses a prefix containing
mathematical functions, constants, and helper functions, along with
the compiled code, to produce the final executable script. This
script is run asynchronously on the HTML canvas element, which
provides all necessary functionality for a basic turtle graphics
renderer.\\\\
Benchmarking our compiler was straightforward. We executed a
predetermined routine of running scripts ten times, calculated
the average execution time and runtime, and computed the standard
deviation. These results were displayed directly in the application.
The benchmarks themselves tested various aspects of the Logo
language, such as loops and recursion.
