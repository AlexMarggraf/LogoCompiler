\section{Abstract}

This report documents the development of a Logo compiling and rendering 
web application and compares different approaches in compiling regarding 
compile time and run time. Furthermore, the web application is also compared 
to the commercial web app XLogoOnline, which allows users to write and run 
Logo code, to understand how compiling versus interpreting code changes the 
performance of the applications.
To achieve this, we built three compilers which used different approaches 
for method calling—namely member expression, computed member expression, 
and hard-coding—and our own renderer, making it accessible through a 
user interface. To test the implementation, we implemented a benchmarking 
functionality and wrote our own benchmarks, designed to test different 
aspects of the programming language.
We successfully managed to build the aforementioned application and found 
that at runtime the three different compilers were roughly equally performant, 
but at compile time the hard-coded approach takes considerably longer. We 
also found that the interpreted approach has performance trade-offs the 
more complex the program is. These findings suggest that our tested compiling 
strategies have small effects on the actual performance of the compiled code, 
and that interpreting might considerably slow down execution time for larger 
programs.