\section{Conclusion}

Our goal was it to write our own web aplication to allow users to write, compile, render and benchmark code,
as well as to test compiling strategies and compare our implementation against the XLogoImplementation.
We successfully managed to develop the afformentioned web application which allows users to select
.lgo files or write their own code, compile the code using direct access, array acces and hard-coded
approaches, run the code and see the result rendered on to the screen as well as benchmark the code
and get the results displayed. Furthermore we found that the different compiling strategies we
implemented performed roughly equaly but the hard-coded approach leads to slower compile times.
Additionally we also found that the XLogoOnline implementation performs roughly equal with our implementation
for smaller programs but struggles to keep up the more complex and long the programs get. These results suggest
that on a large scale it doesn't matter too much whether the code is accessed using member expression or computed
member expression. It also seems as tho as our approach to hard-code the functions instead of accessing them tends
to lead to slightly slower runtimes. Compiling is essentially equaly performant for the array access and direct acces
modes but is roughly twice as slow when sellecting the hard-coded approach. This is probably due to the high ammount
of boilerplate code which needs to be generated as code has to be reparsed using esprima and functions have to be explicitly 
reimplemented before being able to use them. Our findings show that the XLogoOnline implementation struggles to keep
up with our implementation the more complex the programs become. This shows that compiling may lead to higher
performance when it comes to code execution. It has to be said that a fair comparison between 
XLogoOnline and our implementation is not possible. Our application is much less powerfull as 
XLogoOnline provides four different versions of their service and has many features we don't. Furthermore
the renderers being used are also vastly different and will hugely impact performance. We also used a rather small 
ammount of basic benchmarking programs which may impact the generalizability of our evaluation. 