\documentclass[11pt]{article}
\usepackage[utf8]{inputenc}

\usepackage{geometry}
 \geometry{
     a4paper,
     left=25mm,
     top=25mm,
     right=25mm,
     bottom=30mm
 }

%% for colors
\usepackage{xcolor} % must be loaded erly
%% Define vour custom colors
\definecolor{myblue}{HTML}{648FFF}
\definecolor{mypurple}{HTML}{775EF0}
\definecolor{myred}{HTML}{DD2680}
\definecolor{myorange}{HTML}{FE6100}
\definecolor{myyellow}{HTML}{FFB001}
% Create a new command for the colors
\newcommand{\blue}[1]{\textcolor{myblue}{#1}}
\newcommand{\purple}[1]{\textcolor{mypurple}{#1}}
\newcommand{\red}[1]{\textcolor{myred}{#1}}
\newcommand{\orange}[1]{\textcolor{myorange}{#1}}
\newcommand{\yellow}[1]{\textcolor{myyellow}{#1}}
 
%% for tables
% for dashed and dotted lines
\usepackage{tabularray}
\usepackage{booktabs} % For better hlines in tables
% used to add notes to figures and table with fnote command
\newcommand\fnote[1]{\captionsetup{font=small}\caption*{#1}}
% for tables with cells spanning several rows
\usepackage{multirow}
\usepackage{makecell}

%% for grafics and figures
\usepackage{graphicx}
\usepackage{svg}  % to use .svg
\usepackage{tikz}

%% for flexible list items
\usepackage{enumitem}

%% math 
\usepackage{amsmath}
\usepackage{amsthm}
\usepackage{mathtools} 
\usepackage{verbatim} 
\usepackage{todonotes}

%% format of captions and footnotes
\usepackage[font=small,labelfont=bf]{caption}
% flushing left of footnotes
\usepackage[hang,flushmargin]{footmisc}

%% links and urls
\usepackage[hidelinks]{hyperref}
\hypersetup{
    colorlinks=false,
    linkcolor=black,
    filecolor=magenta,      
    urlcolor=blue,
    citecolor=blue,
}
\urlstyle{same}

%% clever referencing and citing rule
\usepackage{cleveref}
\usepackage[sort, numbers]{natbib}
\usepackage{appendix}

%% new setting for \paragraph
\newcommand{\myparagraph}[1]{\paragraph{#1}\mbox{}\\}

%% format of inline code
\definecolor{codeColor}{gray}{0.85}
\usepackage{tcolorbox}
\newtcbox{\ilc}{on line, boxrule=0pt, boxsep=0pt, bottom = 0pt, left=0pt, right =0pt, top = 0pt, arc = 0pt, colback=white, colframe = white, fontupper={\small \ttfamily}}

%% for code snippets
\usepackage{listings}

\title{Logo Compiler}

\author{\\ \\
Project Report\\
ICPL Seminar Spring Semester 2025 \\ \\ \\ \\
University of Basel \\
Faculty of Science \\
Department of Mathematics and Computer Science \\ \\ \\ \\
Alexander Marggraf\\ 
Max Reinert\\
Pascal von Fellenberg\\\\}
\date{June 11, 2025 \vfill \includegraphics{UniBas_Logo_EN_Schwarz_RGB_65.pdf}}

\begin{document}
    \maketitle
    \thispagestyle{empty}

    \clearpage
    \pagestyle{empty}
    \tableofcontents
    
    \clearpage    
    \pagestyle{plain}
    \setcounter{page}{1} % Start from page 1
    \section{Abstract}

This report documents the development of a Logo compiling and rendering 
web application and compares different approaches in compiling regarding 
compile time and run time. Furthermore, the web application is also compared 
to the commercial web app XLogoOnline, which allows users to write and run 
Logo code, to understand how compiling versus interpreting code changes the 
performance of the applications.
To achieve this, we built three compilers which used different approaches 
for method calling—namely member expression, computed member expression, 
and hard-coding—and our own renderer, making it accessible through a 
user interface. To test the implementation, we implemented a benchmarking 
functionality and wrote our own benchmarks, designed to test different 
aspects of the programming language.
We successfully managed to build the aforementioned application and found 
that at runtime the three different compilers were roughly equally performant, 
but at compile time the hard-coded approach takes considerably longer. We 
also found that the interpreted approach has performance trade-offs the 
more complex the program is. These findings suggest that our tested compiling 
strategies have small effects on the actual performance of the compiled code, 
and that interpreting might considerably slow down execution time for larger 
programs.
    \section{Methodology}

Our initial goal was to implement a compiler into the existing XLogoOnline implementation, but we quickly gave up on this as it proved to be too challenging. Instead, we decided to build our own standalone web application that allows the user to write, compile, render, and benchmark code.\\\\
To compile our Logo code, we started off with the raw code itself. Our implementation requires the code to be present in a .lgo file as used by XLogoOnline, or it can be written directly in the application. Files ending in .lgo are essentially .txt files with a different extension. This was convenient as we could easily parse the contents of the files into a string. This string was then parsed into an AST using the grammar, lexer, and parser of XLogoOnline.
With the AST in hand, we used our own compiler to generate a JavaScript AST, which we later used to generate executable code. The compiler iterates over the AST and uses the action set, which holds all the built-in functions of the renderer, to convert the Logo AST nodes into JavaScript AST nodes. However, sometimes additional steps were necessary to ensure correct execution, which will be specified in the Implementation section. Furthermore, the compiler has different modes that determine how the code is generated.
The default mode is direct access. This means that each function call of the action set is done via dot notation. In contrast, the array access mode uses bracket notation to access the functions of the action set, and the direct mode uses no function calls to the action set and instead implements the functions directly inside the code. With the JavaScript AST, we used Escodegen to generate the JavaScript code, which can then be used to render onto the screen.\\\\
To render the code, the application uses a prefix that contains some mathematical functions, constants, and helper functions, as well as the code itself, to generate the final executable script, which is then run asynchronously. For rendering, we used the HTML canvas element, which provides all the necessary functionality for a basic turtle graphics renderer.\\\\
The implemented benchmark functionality is quite straightforward. We generated a predetermined routine of running scripts ten times and calculating the average execution time and runtime, as well as the standard deviation. These results are then displayed directly within the application. The benchmarks themselves are programs that test various aspects of the Logo language, such as loops and recursion.

    \section{Polymorphic Inline Caches}
Our compiler tests various approaches as to how exactly the compiled code should look like. We compare the performance of those results. 
To know why different approaches might behave differently from a performance point of view, we present one possible optimization modern JavaScript-engines employ.
What follows is an introductoin to polymorphic inline caches. 
\subsection{Terminology}
\begin{itemize}
    \item A method-call on an object $o$ is also called a \emph{message send} with \emph{receiver} $o$.
    \item A programming language is AOT-compiled if the code is compiled before the program is run (i.e. at compile time). 
    \item A programming language is JIT-compiled if the code is compiled at runtime.
    \item A place in the code where a message send happens is called a \emph{call site}.
\end{itemize}
\subsection{Overview}
From this point on, we only consider object-oriented languages which are compiled to machine code (either JIT or AOT).
Otherwise the overhead from interpretation of the language would dominate and it would not make sense to optimize the way it is done with inline caches. 

Message sends in object-oriented languages are very expensive if no caching is implemented in the runtime. 
A reason for this is the inheritance rules of those languages that have to be ensured. 
The consequence of this is that before each message send, an expensive lookup procedure must be called which determines, based on the type of the object, which piece of machine code is run. 

The first optimization one can make is to store the results of this lookup procedure in a simple lookup cache (lookup table) which maps the type of the receiver together with the name of the message to the address of the corresponding machine code. 
So at the call site, instead of calling the lookup procedure, the lookup cache is consulted and the expensive lookup procedure is only performed on a cache miss. 
This optimization improves performance considerably. (source???????????????????????????)

However, this lookup cache still implies at least one indirection per message send. 
Inline caching addresses this problem by storing the address of the procedure corresponding to the last message send directly at the call site, together with a check that ensures the type of the object is still the same (note that the message name is assumed to be constant per call site so it doesn't have to be checked). 
When an inline cache miss happens, the correct procedure has to be found - either through the lookup table or the lookup procedure - and it's address inserted at the call site. 
Additionally, the check has to be updated. 
Inline caches work under the assumption that the receiver type of a call site doesn't change often. 

Now what happens if this assumption doesn't work, i.e. the receiver type of a call site changes all the time?
Then we have more overhead compared to the simple lookup table: instead of just consulting the lookup table, we also have to update the call site in each call. 
To circumvent this, polymorphic inline caches do not only store one address in a call site, but several. 
So at each call site, there is conceptually switch statement on the type of the receiver which has call instructions with the procedure corresponding to that type. 
In case of a cache miss (the receiver's type is not in the polymorphic inline cache of this call site), this call site is extended with that type and the corresponding procedure. 

What happens in modern JavaScript engines is that depending on the behaviour of a call site, different methods of caching are employed. 
If the receiver type of a call site doesn't change (a monomorphic call site), it uses an inline cache. 
If the receiver type changes (a polymorphic call site), a polymorphic inline cache is used. 
In call sites where the type changes a lot (megamorphic call sites), more dynamic approaches are needed which are not covered in this overview. 

\subsection{Relevance to our Project}
We implement a compiler which has different modes, varying the way methods are called.
The benchmark shows how the different modes impact performance, i.e. how well the JavaScript engine is able to optimize the compiled code. 


    \section{Logo Overview}
\label{sec:logo-overview}
To know what we had to implement, it is good to have an overview over what constructs the turtle graphics language Logo supports. 
Here are the relevant components.

\subsection{Basic Commands}
\begin{itemize}
    \item Movement of the turtle: \begin{itemize}
        \item \texttt{fd $k$}, \texttt{bk $k$}: go $k$ steps\footnote{number of steps refers to the number of pixels} forward or backward respectively
        \item \texttt{rt $\theta$}, \texttt{lt $\theta$}: turn $\theta$ degrees to the right or left respectively
        \item \texttt{setx $x$}, \texttt{sety $y$}, \texttt{setxy $x$ $y$}: set the position of the turtle or just one component of it
        \item \texttt{home}: go to the origin (same as \texttt{setxy 0 0})
        \item \texttt{setheading $\theta$}: set the direction in which the turtle points, in degrees
    \end{itemize}
    \item State of the turtle: \begin{itemize}
        \item \texttt{setpc $c$}: set the pen color to $c$, where $c$ can be a color by name but also a list of numbers
        \item \texttt{setsc $c$}: set the background color to $c$
        \item \texttt{setpw $w$}: set the width of the pen to $w$
        \item \texttt{pu}, \texttt{pd}: disable or enable the pen, respectively (pen up, pen down)
    \end{itemize}
    \item Clearing the screen: \begin{itemize}
        \item \texttt{wash}: clear the screen but keep the state of the turtle (pen state, position and direction)
        \item \texttt{cs}: clear the screen and the state of the turtle
    \end{itemize}
    \item Miscellaneous: \begin{itemize}
        \item \texttt{print $v$}: print $v$ to the console
        \item \texttt{wait $h$}: wait $h$ hundredths of a second
    \end{itemize}
\end{itemize}

\subsection{Variables}
\begin{itemize} 
    \item \texttt{make "$n$ $v$}: declare variable with name $n$ and initialize it with the value $v$
    \item \texttt{make :$n$ $v$}: assign the value $v$ to the variable with name $n$
\end{itemize}

\subsection{Control Structures}
\begin{itemize} 
    \item \texttt{repeat $k$ [$s$]}: repeat the statements $s$ $k$ times
    \item \texttt{if $c$ [$i$] [$o$]}: if condition $c$ is fulfilled, execute statements $i$ otherwise the statements $o$
    \item \texttt{while [$c$] [$\ldots$]}: repeat the statements inside the square brackets until the condition $c$ is false
    \item \texttt{to $n$ $p$ \\\quad$s$ \\end}: declare a procedure with name $n$ which executes the statements $s$. $p$ is the list of parameters accepted by that procedure which can be used by the statements $s$
\end{itemize}

% TODO maybe add example of all the constructs of Logo

\section{Implementation}
Our implementation consists of a compiler which compiles from Logo to JavaScript. 
The compiled code uses an object which we call action set. 
This action set defines what happens for each basic command in the Logo code. 
This means that for each basic command, a method with the same name exists on the action set which defines it's behaviour. 
In a visual application, such an action set renders the commands in a canvas, but this can be adapted to other circumstances (e.g. for testing). 
Calls on such an action set \texttt{act} can be done in multiple ways in JavaScript, here are the three we decided to test:
\begin{enumerate}
    \item as a member expression: \texttt{act.fd(100)} (in our code, this is called \emph{direct access})
    \item as a computed member expression: \texttt{act['fd'](100)} (in our code, this is called \emph{array access})
    \item or pasting the source code of the method to be called into the call site (in our code, this is called \emph{hard coded})
\end{enumerate}

\subsection{Compiler}
We use the existing parser from the xLogoOnline project to get an abstract representation (abstract syntax tree or AST) of the Logo code.
This AST is then transformed into a JavaScript AST, which is converted back to a string using the library escodegen (source????????????). 
The resulting string can then be run in various JavaScript environments. 

Note that all of the constructs in Logo have matching constructs in javascript. 
This means that our compiler is rather simple in that we mostly just map Logo AST nodes to the corresponding JavaScript AST nodes. 

\subsection{Action Set Calls}
The basic commands are mapped to action set calls in the compiler. How those are generated depends on the strategy the compiler is configured in. 
The cases \emph{direct access} and \emph{array access} are straight forward: 

\begin{tabular}{|l|l|l|}
    \hline
    Logo code & \emph{direct access} & \emph{array access} \\
    \hline
    \texttt{setxy 42 0}& \texttt{act.setxy(42, 0)} & \texttt{act['setxy'](42, 0)}\\
    \hline
\end{tabular}

However, we had to do a bit more to implement the strategy \emph{hard coded}. 
% TODO

\subsection{On wait}
%% TODO

    \section{Performance of Different Approaches}

There are now three different ways to compile the code, the direct approach, the array approach and the hard-coded approach.
One might now wonder which one of these approaches is the fastest one.
To compare these three approaches, multiple benchmarks were introduced in order to test the performance of the approached for different functions.
The first benchmarks were to test the basic functionalities to draw on the canvas and to write simple code.
Here benchmarks for movement, pen manipulations but also repeat blocks, condition blocks and while blocks were implemented.
Since these benchmarks only test the basic functionality, three additional benchmarks to test more applicable cases were added.
So a benchmark for recursions, one for a big picture and a benchmark for an animation were also introduced.
For all the benchmarks, the compile time as well as the runtime was measured.
In order to get a proper measurement, the average over ten executions was calculated since one execution might vary in time.
The result of these benchmarks are shown in the following graphic\ref{fig:compTime}.

\begin{figure}[H]
    \includegraphics[width=\textwidth]{resources/compiletime_shortandmedium_largeandhuge}
    \caption{The measurement of the compile time.}
    \label{fig:compTime}
\end{figure}

In\ref{fig:compTime} the average of the measured compile time for eah benchmark and compiler approach is shown.
It can be seen that for all benchmarks, the hard-coded approach is the slowest one and the direct as well as the array approach gave pretty similar results.
The benchmark for recursion on the other hand also shows a difference in the direct and array approach, here the direct approach is being faster than the array approach.
For the more complex benchmarks like the big picture benchmark and the animation benchmark, we can see that the differences observed before are not as visible anymore.
Although the hard-coded approach is slower for the big picture benchmark and the array approach is faster for the animation benchmark. \newline \newline

After looking at the compile time, the runtime was also measured since the compiled code is different using the three approaches and this might also lead to significant differences.

\begin{figure}[H]
    \includegraphics[width=\textwidth]{resources/runtime_small_medium}
    \caption{The measurement of the run time.}
    \label{fig:runTime}
\end{figure}

In this graph\ref{fig:runTime} the difference in runtime of the code obtained by using the different compiler approaches can be observed.
Here the animation benchmark is missing since the runtime of this benchmark is not as significant as the others and this result is also shown in the comparison to xLogoOnline.
For the actual runtime of the code, it is also obtained that the hard-coded approach leads to an increased runtime.
The direct approach is also almost always faster than the array approach with two exceptions.
The array approach is faster for the while benchmarks and the condition benchmark.
Although only for the while benchmark, the difference in runtime compared to the direct approach is significant. \newline
So it can be said that the hard-coded approach is the slowest approach of these three if the compile time and the runtime are compared.
In general for the benchmarks the direct approach leads to a shorter runtime but longer compile time compared to the array approach. \newline
Whether the direct approach or the array approach is faster, cannot be said since it is a trade-off between compile time and runtime.
If the code has to be compiled multiple times compared to be run, the array approach might be faster but if the code is compiled once and run very often, the direct approach might lead to better results.

    \section{Comparison With XLogoOnline}
Up to this point, only the three compilers were compared.
It is also interesting to compare the results to xLogoOnline, which does not use a compiler but only an interpreter.
Although this makes a comparison between the compiler for Logo and xLogoOnline more complex since we do not have a compiled time and runtime for xLogo.
In order to make a propper comparison, the time it takes for xLogo to go from the input code to the final picture on the canvas is measured.
For the three compiler, the compile time and the runtime were added since this is also the time it takes form the actual code to the picture on the canvas.
Although this is not a real comparison between the interpreter and compilers, it is still possible to make a fair comparison even though the renderers for our implementation nad xLogo are also taken into account.
In order to keep the results pretty visible, we split the benchmarks up into the simpler and faster benchmarks for the basic functionalities and the longer benchmarks which represent larger programs.
The benchmarks are the same as for the comparison between the compilers and there are also shorter and longer benchmarks.

\subsection{Comparison With XLogoOnline Using Shorter Benchmarks}
At first the behavior for the shorter benchmarks was analyzed since most programs containing only basic functions in Logo are kept pretty short.

\begin{figure}[H]
    \includegraphics[width=\textwidth]{resources/endtoend_smallandmedium}
    \caption{The measurement of the benchmarks for simple programs.}
    \label{fig:endToEndShortMedium}
\end{figure}

Here in\Cref{fig:endToEndShortMedium} it can be obtained that the results for xLogo vary in the measured time for each benchmark and the margin of error is big.
But for simpler tasks, it is obtained that the hard-coded approach is still the slowest of them all with xLogo being the second slowest.
Where an actual difference between the direct approach, the array approach and xLogo cna be seen is for the benchmarks for condition blocks and for recursions.
For condition blocks, the xLogo is not much slower than the compiler approached but for the recursion, xLogo is significantly slower than all the compiler approaches.
For recursions xLogo is almost five times slower than the complier approaches. \newline

\subsection{Comparison With XLogoOnline Using Longer Benchmarks}

It is also important to check if this behavior is still visible for longer benchmarks.
So in order to analyze this, the longer benchmarks were also run in XLogo.

\begin{figure}[H]
    \includegraphics[width=\textwidth]{resources/endtoend_smallandmedium_Long}
    \caption{The measurement of the benchmarks for simple programs.}
    \label{fig:endToEndShortMediumLong}
\end{figure}

% TODO Add further analysis here, correct graphic was deleted and will be added soon

\subsection{Comparison With XLogoOnline For Complex and Large Programs}

In the following graph\Cref{fig:endtoEndLargeHuge}, the comparison between the three compiler approaches and xLogo for the big picture benchmark and the animation benchmark can be seen.

\begin{figure}[H]
    \includegraphics[width=\textwidth]{resources/endtoend_large_huge}
    \caption{The measurement of the compile time for more complex programs.}
    \label{fig:endtoEndLargeHuge}
\end{figure}

For both benchmarks, it can be seen that the three compilers perform better than xLogo for complex programs, as already seen for the recursion benchmark.
For big pictures, xLogo is about three to four times slower than the compiled versions.
There is also a difference in the time, it takes xLogo to finish the animations compared to the compilers.
It is important to state tha the time it takes for the animation to finish is heavily dependent on the time it takes to draw each frame.
Since xLogo takes twice as long as the compilers to finish the animation, it can be said that xLogo takes significantly longer to draw each frame of the animation. \newline \newline

So we can say that for simple and small programs, containing almost only basics commands, xLogo is as fast as the direct approach and the array approach.
But if the programs become larger and more complex like adding while loops, condition blocks or adding animations or recursions,
the performance of the compilers becomes better than xLogo.
For the simpler and smaller programs, xLogo is faster than the hard-coded approach but if the programs become more complex, the hard-coded approach also becomes faster than xLogo.
So for realistic scenarios and complex programs with many condition blocks, loops, animations and recursions, the direct and array approach become significantly better and superior compared to xLogo.
    \section{Conclusion}

Our goal was to write our own web application to allow users to write,
compile, render, and benchmark code, as well as to test compiling
strategies and compare our implementation against the XLogo
implementation. We successfully managed to develop the aforementioned
web application, which allows users to select .lgo files or write
their own code, compile the code using direct access, array access, and
hard-coded approaches, run the code, and see the result rendered onto
the screen, as well as benchmark the code and get the results displayed.

Furthermore, we found that the different compiling strategies we
implemented performed roughly equally, but the hard-coded approach led
to slower compile times. Additionally, we also found that the
XLogoOnline implementation performs roughly equal to our implementation
for smaller programs but struggles to keep up as the programs become
more complex and longer.

These results suggest that on a large scale it does not matter too much
whether the code is accessed using member expressions or computed member
expressions. It also seems that our approach of hard-coding the
functions instead of accessing them tends to lead to slightly slower
runtimes. Compiling is essentially equally performant for the array
access and direct access modes, but is roughly twice as slow when
selecting the hard-coded approach. This is probably due to the high
amount of boilerplate code that needs to be generated, as code has to be
reparsed using Esprima and functions have to be explicitly reimplemented
before being able to use them.

Our findings show that the XLogoOnline implementation struggles to keep
up with our implementation as the complexity of the programs increases.
This indicates that compiling may lead to better performance when it
comes to code execution. However, it must be said that a fair comparison
between XLogoOnline and our implementation is not possible. Our
application is much less powerful, as XLogoOnline provides four
different versions of their service and includes many features that we
do not. Furthermore, the renderers being used are also vastly different
and will greatly impact performance. We also used a rather small amount
of basic benchmarking programs, which may impact the generalizability
of our evaluation.
\end{document}